\documentclass[11pt]{article}

\usepackage[utf8]{inputenc}
\usepackage[T1]{fontenc}
\usepackage[francais]{babel}
\usepackage[top=1.8cm, bottom=1.8cm, left=1.8cm, right=1.8cm]{geometry}
\usepackage{hyperref}
\usepackage{graphicx}
\usepackage{amssymb}
\usepackage{epsfig}
\usepackage{array,multirow,makecell}
\hypersetup{
    colorlinks=true,
    breaklinks=true,
    urlcolor=red,
}
\parskip=5pt

\title{Planning}

\begin{document}
\maketitle

\section{Mercredi 31 Janvier}
\begin {itemize}
\item élaboration de l'énoncé du projet Stéganographie \& Stéganalyse
\item apprentissage de la méthode Volere pour la future élaboration du cahier des charges
\item apprentissage de la méthode Gantt pour la gestion du projet et l'organisation d'équipe
\item premières recherches sur le sujet 
\end{itemize}

\section{Mercredi 07 Février}
\begin {itemize}
\item définition des termes du sujet (Préambule)
\item définition de l'historique (Préambule)
\item mise en forme du but du projet (Conducteurs du projet)
\item explication de la motivation du projet (Conducteurs du projet)
\item recherche d'algorithmes de stéganographie/stéganalyse pour le format Image 
\end{itemize}

\section{Mercredi 14 Février}
\begin {itemize}
\item élaboration du calendrier (Contraintes du projet)
\item définition des contraintes imposées (Contraintes du projet)
\item définition de la portée du produit (Exigences fonctionnelles)
\item identification des exigences du client (Exigences fonctionnelles)
\item recherche d'algorithmes de stéganographie/stéganalyse pour le format Audio
\end{itemize}

\section{Mercredi 21 Février}
\begin {itemize}
\item élaboration de l'organigramme des différents modules et recherches des informations qui circulent entre ces modules ; liste des fonctionnalités présentes dans chaque module (Modules du produit)
\item définition de l'apparence et de la perception du produit (Exigences non fonctionnelles)
\item définition de la performance du produit (Exigences non fonctionnelles)
\item identification des exigences culturelles, politiques et légales (Exigences non fonctionnelles)
\item recherche d'algorithmes de stéganographie/stéganalyse pour le format Video
\end{itemize}

\section{Mercredi 23 Février}
\begin {itemize}
\item prise de décision et écriture des algorithmes des fonctionnalités utilisés (Modules du produit)
\item estimations des coûts du produit (Modules du produit)
\item choix du langage et de l'interface (Modules du produit) ; élaboration de la conclusion
\end{itemize}

\section{Elaboration globale du cahier des charges}


\begin{tabular}{|c|c|c|c|}
  \hline
  \textbf{Division du cahier} & \textbf{Sous-Partie du cahier} & \textbf{Fait} & \textbf{Vérifié} \\
   \textbf{des charges} & \textbf{des charges} & \textbf{\checkmark} & \textbf{\checkmark} \\
  \hline
  \multirow{2}*{Préambule} & Définition des termes du sujet & &\\
  \cline{2-4}
   & Historique & &\\
  \hline
   \multirow{2}*{Conducteurs du projet} & But du projet & &\\
   \cline{2-4}
   & Motivation du projet & &\\
  \hline
   \multirow{2}*{Contraintes du projet} & Calendrier & &\\
   \cline{2-4}
   & Contraintes imposées & &\\
  \hline
  \multirow{2}*{Exigences fonctionnelles} & Portée du produit & &\\
  \cline{2-4}  
   & Exigences du client & &\\
  \hline  
    \multirow{3}*{Exigences non fonctionnelles} & Apparence et Perception & &\\
	\cline{2-4}
   & Performance & &\\
  \cline{2-4} 
   & Exigences culturelles, politiques et légales & &\\
  \hline  
  \multirow{3}*{Modules du produit} & Organigramme & &\\
  \cline{2-4}
   & Algorithmes des fonctionnalités & &\\
  \cline{2-4}
  & Estimations des coûts & &\\
  \hline   
  \multirow{4}*{Autres aspects du projet} & Solutions sur étagère déjà existantes & &\\
  \cline{2-4}
   & Tâches à réaliser pour le développement de l'application & &\\
  \cline{2-4}
   & Améliorations pour les versions futures du projet & &\\
  \cline{2-4}  
   & Choix du langage et de l'interface & &\\
  \hline         
  Conclusion & - & &\\
  \hline      
\end{tabular}

\end{document}
