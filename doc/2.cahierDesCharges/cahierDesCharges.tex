\documentclass[11pt]{article}

\usepackage[utf8]{inputenc}
\usepackage[T1]{fontenc}
\usepackage[francais]{babel}
\usepackage[top=1.8cm, bottom=1.8cm, left=1.8cm, right=1.8cm]{geometry}
\usepackage{hyperref}
\hypersetup{
    colorlinks=true,
    breaklinks=true,
    urlcolor=red,
}
\parskip=5pt

\title{Énoncé du projet de Steganogaphie}
\author{Pierre AYOUB, Claire BASKEVITCH, Tristan BESSAC, \\
Clément CAUMES, Damien DELAUNAY, Yassin DOUDOUH}
\date{Mercredi 14 Mars 2018}

\begin{document}

\title{\Huge{\textbf{Cahier Des Charges}}}
	\author{AYOUB Pierre - BASKEVITCH Claire - BESSAC Tristan - \\
		CAUMES Clément - DELAUNAY Damien - DOUDOUH Yassin \\ \\ \\
		Stéganographie \& Stéganalyse \\ \\ \\}
		

	\begin{titlepage}
		\maketitle
		\vspace{20em}
		%\begin{center}\includegraphics{logo.jpg}\end{center}
	\end{titlepage}

\section{Introduction}

\subsection{Définition des termes du sujet}
La stéganographie est l'art de la dissimulation, appliquée en informatique en
cachant des données dans d'autres données. Cette dissimulation se fait
généralement au sein de fichiers multimédias. 

La stéganalyse, quant à elle, est la recherche de données cachées dans des
fichiers suspects. Si ces données sont identifiées, il faut ensuite réussir à les
extraire pour les lire.

\subsection{Historique}
La stéganographie est une méthode très ancienne dont la première référence à cette utilisation date du premier siècle avant Jésus-Christ. 
Elle apparaît dans un récit écrit par Hérodote qui raconte comment deux citoyens communiquaient secrètement : 
le premier citoyen rasait la tête de son esclave et lui écrivait un message sur son crâne. Ensuite, il fallait attendre que les cheveux de l'esclave repousse puis envoyer ce dernier chez le deuxième citoyen. 
Ce dernier devait de nouveau raser la tête de l'esclave pour découvrir le message qui lui était destiné. 
Une autre utilisation de la stéganographie consistait à utiliser de l'encre ,invisible à l'oeil nu, mais qui était révélée à la chaleur. 

Avec l'émergence de l'Informatique, les techniques de Stéganographie se sont renouvelées. En effet, il est désormais possible de cacher des données dans d'autres données. 
Cette multiplicité de techniques stéganographiques grâce à l'Informatique montre l'étendue de cette application dans tous les domaines. 
Par exemple, la stéganographie moderne a été utilisée dans des communications terroristes (transmission de messages) ou dans les signatures de fichiers multimedia (tatouage numérique) afin de protéger les droits d'auteurs. 

\section{Fondements du projet}

Le but du projet est de réaliser un logiciel de Stéganographie. Les utilisateurs ciblés sont des personnes lambdas qui veulent communiquer sans que l'on soupçonne que leurs communications sont en réalité compromettantes. 

Le but de l'application est de permettre à un utilisateur $U_1$ d'envoyer des données cachées à un autre utilisateur $U_2$. Ce deuxième utiliseur devra pouvoir interpréter ces données en utilisant la même application que $U_1$. 


\end{document}

