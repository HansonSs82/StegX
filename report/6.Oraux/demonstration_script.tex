\documentclass[11pt]{article}
%\documentclass{book}
\usepackage[utf8]{inputenc}
\usepackage[T1]{fontenc}
\usepackage[french]{babel}
\usepackage[top=1.8cm, bottom=1.8cm, left=1.8cm, right=1.8cm]{geometry}
\usepackage[linktocpage,colorlinks=false]{hyperref}
\usepackage{graphicx}
\usepackage{epsfig}
\usepackage{amssymb}
\usepackage{amsmath}
\usepackage{array}
\usepackage{subfig}
\usepackage{multicol}
\usepackage{caption}
\usepackage{listings}
\usepackage{algorithm}
\usepackage{algorithmic}
\usepackage{array,multirow,makecell}
\hypersetup{
    colorlinks=true,
    breaklinks=true,
    urlcolor=red,
}
\parskip=5pt

\title{\huge{\textbf Démonstration}}
\author{AYOUB Pierre, BASKEVITCH Claire, BESSAC Tristan, \\
CAUMES Clément, DELAUNAY Damien, DOUDOUH Yassin}
\date{Vendredi 1 Juin 2018}

\begin{document}

\maketitle
\vspace{20em}
\newpage

\tableofcontents

\newpage

\section{Introduction et demande du client}

\subsection{Définitions des termes du sujet}

La stéganographie est l'art de la dissimulation, appliquée en informatique en
cachant des données dans d'autres données. Cette dissimulation se fait
généralement au sein de fichiers multimédias. La stéganographie se différencie
de la cryptographie, qui correspond à chiffrer un message afin qu'il soit
illisible par une personne différente de l'émetteur et du destinataire. En
effet, un message chiffré en cryptographie sera visible par tous mais illisible,
tandis qu'un message caché dans un fichier $f$ en stéganographie ne sera vu que
si un inconnu sait que $f$ contient un message et connaît l'algorithme pour
l'interpréter. 

La stéganalyse, quant à elle, est la recherche de données cachées dans des
fichiers suspects. Si ces données sont identifiées, il faut ensuite réussir à
les extraire pour les lire. Il s'agit donc de la méthode inverse à la
stéganographie. 

\subsection{Commande du client}

Le client veut une application qui suit les caractéristiques suivantes : 
\begin{itemize}
    \item L'application doit permettre de cacher des données dans des fichiers
        de type image, audio et vidéo. 
        Le programme fera l'extraction automatique des données cachées du 
        fichier à analyser.
    \item Plusieurs formats de fichier devront être gérés par l'application,
        ainsi qu'une diversité dans les algorithmes proposés.
    \item L'application se présentera sous forme d'une bibliothèque partagée par
        deux interfaces différentes, une graphique et une en ligne de commande.
\end{itemize}

Le but du projet était donc de réaliser un logiciel de stéganographie permettant à des
personnes lambdas de communiquer sans que l'on soupçonne que leurs
communications soient en réalité compromettantes. 

\section{Enchainement des différents modules}

\begin{description}
\item[a)] \textbf{Interfaces} :
    interfaces permettant à l'utilisateur de choisir parmi les deux
    fonctionnalités possibles de l'application. Il peut dissimuler des
    données dans un fichier (dont le type et le format sont pris en charge
    par l'application). Ou bien, il peut extraire les données cachées d'un
    fichier. \newline
    Il aura donc un mécanisme pour choisir le fichier hôte et le fichier 
    à cacher (pour la dissimulation de données), et un mécanisme pour choisir le 
    fichier contenant les données cachées à analyser (pour l'extraction de 
    données cachées), grâce à une interaction avec le gestionnaire d’entrées/sorties. 

\item[b)] \textbf{Vérification de la compatibilité des fichiers} : le format 
	du fichier hôte (pour le module \textit{Dissimulation de données}) ou 
	le format du fichier à analyser (pour le module \textit{Extraction de données 
	cachées}), choisis par l'utilisateur, sont vérifiés pour savoir s'ils sont
	bien pris en charge par l'application. \newline
	Il aura un mécanisme de lecture du fichier hôte : selon les formats 
	pris en charge, il y aura une série de tests successifs pour déterminer le 
	format du fichier. 

\item[c)] \textbf{Proposition des algorithmes de stéganographie} : en fonction
    du type et du format du fichier hôte, ainsi que de la taille des données à
    cacher, un mécanisme proposera un ou plusieurs algorithmes (avec ou sans 
    sécurité supplémentaire).  

\item[d)] \textbf{Détection de l'algorithme de stéganographie} : un mécanisme 
	d'analyse du fichier permettra de découvrir quel algorithme a été utilisé 
	afin de les extraire correctement par la suite. 

\item[e)] \textbf{Insertion des données} : la copie des données du fichier hôte
    sera modifiée avec l'insertion des données à cacher à l'aide de l'algorithme
    choisi par l'utilisateur. 

\item[f)] \textbf{Extraction} : les données cachées dans le fichier à analyser
    sont extraites selon l'algorithme déduit.

\end{description}

\section{Algorithmes de stéganographie}

\subsection{Least Significant Bit}

Pour l'insertion des données, l'algorithme LSB va modifier les bits de 
poids faible des différents octets des données de l'hôte. Selon la taille des
données à cacher et la taille du fichier hôte, les données cachées pourront 
être dispersées dans les octets de l'hôte. Sinon, on exercera un XOR avec 
la suite pseudo-aléatoire générée à partir du mot de passe. 
Ensuite, quand toutes les données de l'hôte modifiées par LSB ont été écrites 
dans le fichier résultat, la signature StegX est écrite. 

Pour l'extraction des données, la lecture de la fin du fichier permettra de 
lire la signature StegX. Ensuite, les octets de l'hôte seront lues afin d'extraire 
les données cachées. 

Il sera proposé pour les formats BMP (non compressé) et WAV (PCM).

\subsection{End Of File}

Pour l'insertion des données, cet algorithme consiste à recopier le fichier 
hôte dans le fichier résultat, suivi de la signature, suivi des octets 
représentant le fichier à cacher. Cet algorithme fonctionne car les éditeurs 
des formats pour lesquels l'algorithme est proposé n'interprètent pas les données 
après la fin << officielle >> du fichier. 

Pour l'extraction des données, il faut aller à la fin officielle du fichier 
à analyser. A cet endroit précis est écrit la signature StegX. Ainsi, 
les informations sur le fichier caché sont interprétées. Puis, il y a une 
lecture après cette signature, correspondant aux données cachées. 

Il sera proposé pour les formats BMP, PNG, MP3, WAV et FLV.

\subsection{Metadata}

Pour chaque format pris en charge par StegX qui propose cet algorithme, 
ce dernier diffère. En effet, l'étude approfondie du format du fichier est 
nécessaire pour connaître les endroits du fichier où cacher des métadonnées. 
Cela sera dans ces espaces que les données seront cachées. 

Pour l'insertion, il est nécessaire que la signature StegX soit écrite à 
la fin du fichier afin que l'extraction puisse se dérouler correctement. 

Il sera proposé pour les formats BMP et PNG. 

\subsection{End Of Chunk}

Cet algorithme est uniquement proposé par le format FLV et consiste à 
écrire après les différents chunks du fichier hôte. En effet, lors de 
son interprétation, les chunks non reconnus sont ignorés et ce sont à ces 
endroits que les données seront cachées. 
Les données cachées sont situées après chaque chunk vidéo et sont interprétés 
mais ne modifient pas l'image de la vidéo. 

\subsection{Junk Chunk}

L'algorithme Junk Chunk est seulement proposé par le format AVI. Il correspond 
à la création d'un chunk poubelle. Ce chunk est spécifique au format car il
représente un chunk dans lequel les données ne seront pas interprétées par 
les logiciels multimédia. 

\section{Conclusion}

Le produit répond bien aux objectifs et propose ces formats avec ces 
algorithmes : 
\newline

\begin{tabular}{|c|c|c|c|c|c|}
  \hline
  \multirow{2}*{\textbf{Format}} & \multicolumn{5}{c|}{\textbf{Algorithmes proposés}} \\
   \cline{2-6}
    & \textbf{LSB} & \textbf{EOF} & \textbf{Metadata} 
    &\textbf{EOC} & \textbf{Junk Chunk} \\
  \hline
  \textbf{BMP} & \textbf{\checkmark} & \textbf{\checkmark} & \textbf{\checkmark} &  & \\
  \hline      
  \textbf{PNG} &  & \textbf{\checkmark} & \textbf{\checkmark} & & \\
  \hline
  \textbf{WAV} & \textbf{\checkmark} & \textbf{\checkmark} & & & \\
  \hline 
  \textbf{MP3} & & \textbf{\checkmark} & & & \\
  \hline 
  \textbf{AVI} & & & & & \textbf{\checkmark}\\
  \hline
  \textbf{FLV} & & \textbf{\checkmark} & & \textbf{\checkmark} & \\
  \hline
\end{tabular}
\vspace{0.5cm}

Un des avantages de notre logiciel est le fait qu'il peut être constamment améliorer,
du fait que nous avons fait en sortie d'avoir une conception flexible et
modulaire permettant de rajouter facilement de nouveaux formats pris en charges,
ou bien de nouveau algorithmes.

\end{document}
