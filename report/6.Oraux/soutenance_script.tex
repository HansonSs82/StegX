% This file is part of the StegX project.
% Copyright (C) 2018  StegX Team
% 
% StegX is free software: you can redistribute it and/or modify
% it under the terms of the GNU General Public License as published by
% the Free Software Foundation, either version 3 of the License, or
% (at your option) any later version.
% 
% This program is distributed in the hope that it will be useful,
% but WITHOUT ANY WARRANTY; without even the implied warranty of
% MERCHANTABILITY or FITNESS FOR A PARTICULAR PURPOSE.  See the
% GNU General Public License for more details.
% 
% You should have received a copy of the GNU General Public License
% along with this program.  If not, see <https://www.gnu.org/licenses/>.

\documentclass[11pt]{article}
%\documentclass{book}
\usepackage[utf8]{inputenc}
\usepackage[T1]{fontenc}
\usepackage[french]{babel}
\usepackage[top=1.8cm, bottom=1.8cm, left=1.8cm, right=1.8cm]{geometry}
\usepackage[linktocpage,colorlinks=false]{hyperref}
\usepackage{graphicx}
\usepackage{epsfig}
\usepackage{amssymb}
\usepackage{amsmath}
\usepackage{array}
\usepackage{subfig}
\usepackage{multicol}
\usepackage{caption}
\usepackage{algorithm}
\usepackage{algorithmic}
\hypersetup{
    colorlinks=true,
    breaklinks=true,
    urlcolor=red,
}
\parskip=5pt

\title{\huge{\textbf Soutenance orale}}
\author{AYOUB Pierre, BASKEVITCH Claire, BESSAC Tristan, \\
CAUMES Clément, DELAUNAY Damien, DOUDOUH Yassin}
\date{Vendredi 1 Juin 2018}

\begin{document}

\maketitle
\vspace{20em}
\newpage

\tableofcontents

\newpage

\section{Introduction}

\subsection{Commande du client}

Pour le projet S6 de Licence informatique, un client nous a demandé de 
réaliser un logiciel de stéganographie permettant à des
personnes lambdas de communiquer sans que l'on soupçonne que leurs
communications soient en réalité compromettantes. 
Il voulait aussi que l'application propose plusieurs algorithmes, sur 
plusieurs formats différents. 
   

\subsection{Définitions des termes du sujet}

La stéganographie est l'art de la dissimulation, appliquée en informatique en
cachant des données dans d'autres données. Cette dissimulation se fait
généralement au sein de fichiers multimédias. La stéganographie se différencie
de la cryptographie, qui correspond à chiffrer un message afin qu'il soit
illisible par une personne différente de l'émetteur et du destinataire. 

La stéganalyse, quant à elle, est la recherche de données cachées dans des
fichiers suspects et l'extraction. Il s'agit donc de la méthode inverse à la
stéganographie. 

Le but du projet était de réaliser un logiciel de stéganographie permettant à des
personnes lambdas de communiquer sans que l'on soupçonne que leurs
communications soient en réalité compromettantes. 

\section{Description générale des modules}

L'étude des différentes propriétés de la stéganographie et de la stéganalyse 
nous a mené vers cet organigramme. 
Il permet de visualiser les étapes successives de l'insertion des données ainsi
que celle de la l'extraction des données.

Pour la dissimulation de données, il va d'abord y avoir la \textit{Vérification 
de la compatibilité} (b) du fichier hôte. 
Il y a ensuite le module \textit{Proposition des algorithmes de stéganographie} 
(c) qui va être utilisé.
La dernière étape consiste en la réelle insertion des données, où les 
données à cacher seront dissimulées dans une copie du fichier hôte (e). 

Pour l'extraction de données, le module \textit{Vérification de la compatibilité 
des fichiers} aura la même fonctionnalité que pour l'insertion.
Il y aura ensuite une \textit{Détection de l'algorithme de stéganographie} 
(d). 
Enfin, l'\textit{Extraction} (f) permettra de récupérer les données cachées 
dans le fichier à analyser. 

Tous ces modules et sous-modules seront coordonnées grâce aux interfaces 
graphique et en ligne de commande (a), proposées par StegX. 

\section{Présentation du fonctionnement}

Nous allons maintenant décrire le fonctionnement interne de notre
application en illustrant par un exemple de dissimulation et d'extraction 
de données, correspondant à une communication entre Alice et Bob, surveillée 
par Oscar. Les données sont issues d'une réelle utilisation de StegX. 

Alice choisit de dissimuler le message dans une image qu'elle veut envoyer 
à Bob à l'aide de l'interface graphique. 
Dans cet exemple : 
\begin{itemize}
\item le fichier à cacher est message.txt. 
\item le fichier hôte est photo.bmp. 
\item le chemin du fichier à créer est piece\_jointe.bmp
\item le mot de passe est "alicebob" (communiqué à Bob sur un canal sûr).
\end{itemize}

\section{Algorithmique \& Stéganographie}

\subsection{Présentation}

L'application de stéganographie va proposer à l'utilisateur une suite 
d'algorithmes (selon le format du fichier hôte) : 
LSB, EOF, Metadata, EOC, Junk Chunk. 

De plus, pour chaque algorithme, on utilise 2 méthodes différentes selon 
la taille du fichier hôte et/ou du fichier à cacher. 

Le grand nombre d'algorithmes ainsi que la nécessité d'obtenir certaines 
informations pour l'extraction impose la création d'une signature. 

\subsection{Description de la signature StegX}

La signature de StegX est primordiale dans l'insertion puisqu'elle permet au 
destinataire de pouvoir interpréter ce que l'émetteur a voulu lui communiquer. 
Pour cela, cette signature doit avoir plusieurs champs : 
\begin {enumerate}
\item Identificateur de la méthode (1 octet). Cet octet ne 
peut prendre que deux valeurs possibles : la méthode avec ou sans mot de 
passe. Si l'émetteur choisit un mot de passe, StegX récupèrera ce mot de 
passe pour réaliser l'insertion de façon sécurisée. Le destinataire devra 
choisir le même mot de passe pour extraire les données correctement. 
Si par contre, l'émetteur ne choisit pas de mot de passe, StegX va en choisir 
un au hasard et faire les mêmes manipulations qu'avec un mot de passe. 
Le mot de passe par défaut devra donc être écrit dans la signature afin que 
le destinataire puisse extraire. 
\item Identificateur de l'algorithme (1 octet) : cette variable représente 
l'algorithme utilisé par l'émetteur pour que le récepteur extrait correctement 
les données. 
\item Taille du fichier caché (4 octets) : ce nombre représente en octets 
la taille du fichier caché dans le fichier hôte. Il a une taille maximale 
de $2^{32}$ octets, soit 4 GB. Ce nombre est écrit en little endian. 
\item Taille du nom du fichier caché (1 octet) : cet octet représente le 
nombre de caractères du nom du fichier caché sans compter le caractère de 
fin '$\backslash$0'.
\item Nom du fichier caché (entre 1 et 255 octets) : représente le nom du 
fichier caché. Ce nom va être XOR avec le mot de passe (par défaut ou choisi
par l'utilisateur) : ainsi, si le récepteur choisit un mot de passe différent, 
il ne pourra même pas connaître le nom du fichier caché. 
\item Mot de passe (64 octets) : représente le mot de passe par défaut choisi 
aléatoirement par l'application. Si la méthode est avec mot de passe, 
ce champ n'existe pas car c'est le destinataire qui doit le connaître et 
non l'application qui doit l'extraire. 
\end{enumerate}

\subsection{Least Significant Bit}

Pour l'insertion des données, l'algorithme LSB va modifier les bits de 
poids faible des différents octets des données de l'hôte. Selon la taille des
données à cacher et la taille du fichier hôte, les données cachées pourront 
être dispersées dans les octets de l'hôte. Sinon, on exercera un XOR avec 
la suite pseudo-aléatoire générée à partir du mot de passe. 

Il sera proposé pour les formats BMP (non compressé) et WAV (PCM).

L'avantage de cet algorithme est le fait que cela n'augmente pas la 
taille du fichier hôte (à part pour l'ajout de la signature). 
L'inconvénient est que le fait que le fichier a caché doit être assez petit 
pour pouvoir le cacher intégralement dans l'hôte. De plus, la modification 
des bits de poids faible peut être vu (BMP) ou entendu (WAV) par l'homme. 

\subsection{End Of File}

Pour l'insertion des données, cet algorithme consiste à recopier le fichier 
hôte dans le fichier résultat, suivi des octets représentant le fichier 
à cacher. Cet algorithme fonctionne car les éditeurs 
des formats pour lesquels l'algorithme est proposé n'interprètent pas les 
données après la fin << officielle >> du fichier. 

Pour l'extraction des données, il faut aller à la fin officielle du fichier 
à analyser pour lire les données cachées. 

Il sera proposé pour les formats BMP, PNG, MP3, WAV et FLV.

L'avantage de cet algorithme est le fait qu'il n'y a pas de limite de 
taille pour le fichier à cacher. L'inconvénient est le fait que la taille 
du fichier résultat sera égale à la taille de l'hôte, plus la taille de la 
signature, plus la taille du fichier à cacher. 

\subsection{Metadata}

Pour chaque format pris en charge par StegX qui propose cet algorithme, 
ce dernier diffère. En effet, l'étude approfondie du format du fichier est 
nécessaire pour connaître les endroits du fichier où cacher des métadonnées. 
Cela sera dans ces espaces que les données seront cachées. 

Pour l'insertion, il est nécessaire que la signature StegX soit écrite à 
la fin du fichier afin que l'extraction puisse se dérouler correctement. 

L'avantage de cet algorithme est le fait qu'il n'y a pas de limite de 
taille pour le fichier à cacher. L'inconvénient est le fait qu'il augmente 
(comme pour EOF) la taille du fichier hôte.  

Il sera proposé pour les formats BMP et PNG. 

\subsection{End Of Chunk}

Cet algorithme est uniquement proposé par le format FLV et consiste à 
écrire après les différents chunks du fichier hôte. En effet, lors de 
son interprétation, cela ne modifiera pas la vidéo d'origine. 

L'avantage de cet algorithme est le fait qu'il n'y a pas de limite de 
taille pour le fichier à cacher. Par contre, l'inconvénient de cet algorithme 
est le fait que cela peut créer des distorsions sur le fichier vidéo où 
l'interprétation pour certains logiciels multimédia peut être fait sur ces 
données cachées. 

\subsection{Junk Chunk}

L'algorithme Junk Chunk est seulement proposé par le format AVI. Il correspond 
à la création d'un chunk poubelle. Ce chunk est spécifique au format car il
représente un chunk dans lequel les données ne seront pas interprétées par 
les logiciels multimédia. 

L'avantage de cet algorithme est le fait qu'il n'y a pas de limite de 
taille pour le fichier à cacher.  L'inconvénient est le fait qu'il augmente 
(comme pour EOF) la taille du fichier hôte.  

\section{Description des modules}

\subsection{Vérification de la compatibilité des fichiers}

Ce module est commun à l'insertion et l'extraction. Il correspond à la 
recherche détaillée de la nature du fichier hôte. 
Pour cela, il va y avoir une série de tests durant lesquels les
signatures des différents formats proposés par l'application vont être recherchées. 
Si l'une d'elle est reconnue, cela signifie qu'il s'agit du format en question. 
A la fin de la série de tests, si le format n'a pas été reconnu, 
l'application renvoie une erreur : en effet, elle ne peut pas utiliser ce 
fichier là pour dissimuler ou extraire des données. 

Dans notre exemple, il s'agit d'un fichier BMP non compressé dont les 
pixels sont codés sur 24 bits (8 bits par composantes couleurs). 

\subsection{Proposition des algorithmes de stéganographie}

\begin{itemize}
\item Dans ce module, l'application va lire en détails les particularités du fichier 
hôte. En effet, dans le module précédent, nous avons trouvé le format. Il 
faut maintenant lire le header du fichier pour l'interpréter correctement 
par la suite. Lorsque les données spécifiques du fichier ont été trouvées, 
une série de tests sur les différents algorithmes proposés par l'application
permettra de déterminer quels algorithmes sont possibles avec la nature 
du fichier hôte et la taille du fichier à cacher. 
\newline 
La structure spécifique \textit{infos.host.file\_info.bmp} sera remplie, 
notamment avec la taille du header, la taille du bloc data, la longueur 
des pixels et le nombre de pixels dans l'image. De cette structure
est déduit les algorithmes proposés (EOF, LSB et Metadata). 
\newline
\item Après cette série de tests, la récupération du choix de l'utilisateur 
permettra de déterminer son choix d'algorithme à utiliser lors de l'insertion. 
\newline 
Dans cet exemple, Alice choisit l'algorithme EOF. 
\end{itemize}

\subsection{Insertion des données}

Lorsque le fichier à cacher, l'algorithme de stéganographie à utiliser, 
la nature du fichier hôte, le mot de passe (s'il y en a un) et l'emplacement 
du fichier à créer sont connus, il est temps de réaliser la dissimulation. 
Chaque algorithme diffère plus ou moins pour chaque format. 
L'étape commune est le fait que si le fichier à cacher est trop grand, 
l'utilisation du mot de passe permettra de générer (à partir d'une graine), 
une suite de nombres pseudo-aléatoires afin de faire un XOR avec chaque 
octet du fichier à cacher. Si la taille le permet, les octets à cacher 
seront mélangés (grâce au mot de passe) afin de rendre impossible l'extraction 
par une personne inconnue. 

Dans notre exemple, suivant l'algorithme EOF, il va y avoir une écriture 
dans piece\_jointe.bmp du fichier hôte. 
Ensuite, la signature StegX est écrite. 
Enfin, les données cachées sont également écrites à l'aide du mot de passe
grâce à l'algorithme de protection des données. 

Alice pourra donc envoyer le fichier piece\_jointe.bmp sur un canal non 
sécurisé. 

Oscar, qui espionne les communications entre Bob et Alice, n'aura aucun 
soupçon en voyant piece\_jointe.bmp qui semble être une image comme une 
autre. 

\subsection{Détection de l'algorithme de stéganographie}

Ce module correspond à la lecture de la signature StegX que nous expliquerons 
par la suite. En fonction du format
du fichier à analyser, la position de la signature de StegX dans le fichier ne
sera pas la même. Cette lecture sera déterminante pour découvrir l'algorithme
utilisé, le nom du fichier caché et la taille de ce dernier. 

Lors de la détection de l'algorithme, Bob déduira les mêmes champs de 
\textit{infos.host.file\_info.bmp} que Alice lors de l'insertion. En effet, 
il va y avoir une lecture approfondie du fichier. Ensuite, la signature 
StegX, qui a été écrite lors de l'insertion, sera lue.

\subsection{Extraction des données}

De la même manière que l'insertion, 
après une vérification de la compatibilité et une détection de l'algorithme 
de stéganographie, le nom du fichier caché, la taille du fichier caché, 
l'algorithme, le mot de passe (s'il y en a un) et l'emplacement du fichier 
prochainement extrait sont connus, l'extraction peut commencer. 
Le mot de passe permettra de retrouver la même suite de nombres pseudo-aléatoires 
pour XOR avec les octets extraits (si le fichier caché est trop gros). 
Si sa taille le permet, le fichier caché sera créé en remettant dans l'ordre 
les octets extraits. 

Dans notre exemple, les données cachées seront extraites grâce au mot de passe 
"alicebob" choisi par Alice lors de l'insertion et nécessaire pour l'extraction 
effectuée par Bob. 
Les données cachées extraites seront écrites dans le fichier
/user/home/Bureau/message.txt. Bob pourra ainsi lire le message de Alice 
contenu dans le fichier message.txt. 

\section{Problèmes et points délicats rencontrés}

\subsection{Lecture et écriture de fichiers, endianness}

La première difficulté est arrivée lors de l'implémentation du module 
\textit{Vérification de la compatibilité des fichiers}. En effet, il fallait 
lire la signature des fichiers hôtes pour l'insertion et l'extraction. 

Nous ne comprenions pas pourquoi, sur certains formats, le little endian était
utilisé tandis que sur d'autres, le big endian était utilisé. De plus, il
fallait également convertir les octets lus dans l'endian de la machine de
l'utilisateur. L'équipe de conception a fait des recherches à propos des
processeurs little/big endian et les processeurs fonctionnant en big endian sont
principalement des processeurs pour les systèmes embarqués. Dans le cas des
ordinateurs de bureau fonctionnement avec des processeur x86, le little endian
est utilisé. De ce fait, il fallait lire les données des fichiers et les
convertir en little endian. 

Nous avons donc trouvé une solution : la création de fonctions de conversion
entre les deux endianness. L'équipe StegX pouvait utiliser la les fonctions
fournies dans le header \textit{endian.h}, conforme à la norme POSIX du C.
Cependant, notre application étant multi-plateforme (Linux et Windows), il nous
fallait utiliser des fonctions conformes au C ANSI/ISO. Le meilleur moyen était
donc de reproduire nous même ces fonctions de conversion pour l'application 
\label{endian}. 

\subsection{Étude des différents formats et algorithme Metadata}

La deuxième difficulté rencontrée est celle de l'implémentation de l'algorithme
Metadata. La particularité de cet algorithme est le fait qu'elle dépend du
format du fichier hôte. Il fallait donc étudier avec précision la nature 
des éléments qui composent le format ciblé.  

\subsection{Format MP3}

Le format MP3 est sans nul doute le format le plus compliqué de ceux que
l'équipe StegX s'est occupé. En effet, il s'agit d'un format compressé utilisant
des algorithmes de compression divers et complexes. Pour ce format, il a fallu
que le binôme composé de Pierre Ayoub et Damien Delaunay, chargé de réaliser
l'insertion et l'extraction des formats audio, décortique les spécifications
précises de plusieurs versions du format (MPEG 1 Layer III, MPEG 2 Layer III),
et de plusieurs versions de formats de métadonnée (ID3 version 1 et ID3 version
2). Une fois cela fait, il a fallu récupérer plusieurs fichiers échantillons
correspondant aux différents formats et versions afin de les analyser en
hexadécimal et en faire la lecture pour en comprendre le mécanisme. Ces
opérations ont été longues et pointilleuses et aucun algorithme ne pouvait être
proposé dans la version de StegX rendue pour le 25 mai. 

\section{Bilan logiciel et humain}

\subsection{Bilan logiciel}

Le produit StegX répond bien aux objectifs : faire de la stéganographie sur des
fichiers de type image, audio et vidéo en utilisant plusieurs algorithmes. Par
ailleurs, StegX propose bien 2 interfaces : une en ligne de commande et une
autre graphique. Avoir choisi le langage C a été le meilleur choix car ce
langage nous a permis de manipuler facilement les données binaires des fichiers
en entrée. Elle propose également pour les futurs développeurs une librairie 
afin de faire de la stéganographie. 

En plus de répondre aux objectifs, l'équipe de conception s'est efforcée 
de répondre aux besoins du client. 

\subsection{Bilan humain}

En ce qui concerne le bilan humain de ce projet, tous les membres de l'équipe 
de conception se sont efforcés à travailler de façon professionnelle et ordonnée. 
En effet, ils leur tenaient à coeur de mettre en application toute la méthodologie 
informatique acquise durant la licence. Par ailleurs, en raison de la grande 
charge de travail, il a fallu dès le début du projet diviser la conception 
selon les trois différents types dont StegX doit se charger.

Enfin, nous n'avions pas de chef de projet. Cette décision nous a été très
favorable du fait que chacun a eu la maturité de comprendre les enjeux de ce
grand projet. En effet, ces enjeux étaient grands et c'est pour cela que le
projet IN608 est la matière la plus travaillée du semestre. Cela aura été le
projet le plus apprécié de la licence par toute l'équipe, car il était de très
grande envergure et l'équipe a proposé son propre sujet. 

\end{document}
